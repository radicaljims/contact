\documentclass{article}

\usepackage{listings, amssymb, amsmath, tikz-cd, setspace}

\usepackage{hyperref}
\hypersetup{pdftex,colorlinks=true,allcolors=blue}
\usepackage{hypcap}

\begin{document}

\title {Contact Geometry}
\maketitle

\centerline{\sc \large A Very Small Introduction }
% \vspace{.5pc}
% \vspace{2pc}
\onehalfspace

\tableofcontents

\newpage

\textbf{Emphatic note}: none of the mathematical ideas here represent original work in
either content or presentation! The main sources for the material are McInerney
\cite{mcinerney} and Bachman \cite{bachman} but see the full list of references for other very nice resources.

\newpage

\section {Just The Facts}

What I want to get across:

What are the core ideas:

* Fields of planes (actually fields of hyperplanes (co-dimension 1))
* One-forms (and their kernels) can represent fields of planes
* We also want a non-degeneracy / non-integrability condition on the contact
form ($w \wedge dw \neq 0$)

Where do you see these ideas:

* A few places in physics: geometric optics (wavefronts), thermodynamics (the
first law here can be written as a contact form?), classical mechanics (constant
energy hypersurface of a mechanical system)
* PDEs - the theory partly stems from Sophus Lie's work on ``contact transformations''
* Tied up with Legendre transform (that switches between Lagrangian and
Hamiltonian formulations of CM)
* Hamiltonians are everywhere in the contact literature!
* Big resurgence in the '80s thanks to VI Arnold (first place I ever heard of
contact geometry was his book Mathematical Methods of Classical Mechanics!)
* According to Arnold: ``Contact geometry is all of geometry''

Examples of plane fields:

* try to draw something, but probably just reference the picture everyone uses
* talk about foliations / integral curves - we've seen a few of these,
especially on geodesically complete surfaces
* Frobenius integrability
* but we don't want an integral surface and sadly I don't quite understand why

Examples of differential forms:

* Talk about covectors

We've talked about at least three different forms in class: the first, second,
and third fundamental forms.

They are called \textit{forms} because they are functions that map vectors to a
number.

The fundamental forms are also \textit{bilinear}, but there are also simpler
\textit{linear} forms. We will call these \textbf{one-forms}.

If you've taken enough linear algebra you may recognize one-forms as living in
the \textit{dual space} $\mathbb{V}^{*}$ associated to a vector space $\mathbb{V}$.

If you've taken enough physics you may recognize these as \textit{covectors}.

From a geometrical perspective a very important fact about covectors is that
they have a natural pairing with vectors.

If $v \in \mathbb{V}$ and $w \in \mathbb{V}^{*}$ then the number $w(v) \in
\mathbb{R}$ is \textbf{invariant}: it does not depend on what basis $v$ or $w$
may be expressed in.

Said another way the pairing does not depend on coordinates!

Avoiding coordinate representations as much as possible is a big goal of this
formalism. Nonetheless I am about to introduce bases for both $\mathbb{V}$ and $\mathbb{V}^{*}$!

For a given basis $e_{i}$ of $\mathbb{V}$ there is an associated basis $e^{j}$
of $\mathbb{V}^{*}$ called the \textbf{dual basis}.

The dual basis has the property that under the pairing we have $e^{j}(e_{i} = \delta^{j}_{i})$.

That is the \textbf{Kronecker delta} which is $1$ when $i = j$ and $0$
otherwise.

If we then write a $v \in \mathbb{R}^{2}$ as $v = ae_{1} + be_{2}$ and apply the
dual form $e^{1}$ we get $e^{1} (v) = e^{1}(ae_{1} + be_{2}) = e^{1}(ae_{1}) +
  e^{1}(be_{2}) = a(e^{1}(e_{1})) + b(e^{1}(e_{2})) = a(1) + b(0) = a$

Which shows that elements of the dual basis are projection operators.

If we specialize to the context of differential geometry then our $\mathbb{V}$
becomes a tangent plane or space, maybe $T_{p}\mathbb{R}^{3}$.

The dual vector space $\mathbb{V}^{*}$ then becomes the \textbf{cotangent space}
(TODO what's notation for this, do I still use T?).

In class we're used to writing $\rho_{u}, \rho_{v}$ or maybe $r_{u}, r_{v}$ as a
basis for the tanget space.

In these notes we're going to try to use $\frac{\partial}{\partial x}, \frac{\partial}{\partial
  y}$.

For the cotangent space we'll use $dx, dy$.

* One-forms, two-forms, three-forms

Differential forms are grounded in \textit{multilinear} algebra, so we'll be
interested in \textit{multilinear} functions. But a very specific kind of
multilinear function - antisymmetric (or sometimes skew-symmetric) ones.

We actually already know an antisymmetric multilinear function and that is the
cross product!

That's going to be handy because one way to motivate the idea of (not yet
differential) forms is to try to generalize the cross product to higher
dimensions.

It turns out you can't, exactly, but you can get something even cooler: the
\textbf{wedge} (or exterior) product.

Why is it cooler? It exists for every dimension and it lets you build linear
subspaces of arbitrary rank in the same way that a cross product kind of
represents a plane.

In fact the cross product of two vectors in $\mathbb{R}^{3}$
is dual in a sense to the wedge product of two vectors. The cross product being
the normal associated to the plane while the wedge product represents the subset
of the plane spanned by the parallelogram formed by the vectors.

That was a mouthful but the picture looks like this:

TODO INSERT PICTURE

* Organize these into the algebra of three-forms
* Exterior derivative
* Do a quick example of integrating them

Examples of contact forms:

* The ``standard'' example: $xdy + dz$
* Maybe something in polar form
* Talk about how contact forms don't *uniquely* determine plane fields
* Talk about Darboux - every contact form ``locally'' looks the same but is
globally different (reflected in Reeb flow?)

Examples of contact structures:

* line segments in the plane
* planes in space
* the mechanical example from that paper

Examples of Reeb fields:

* talk about flows a bit
* talk about what characterizes the Reeb field / flow
* re: Huygens, talk about how the Reeb flow is maybe light rays? And there's
some relationship to geodesic flow
* talk about how contact forms that generate the same plane fields may *not*
generate the same Reeb fields (I think)

\section {Motivation?}

\subsection {Roots}

\paragraph{``Contact geometry is all of geometry''} the Russian mathematican
V.I. Arnold is reputed to have said.

He was one of the greatest geometers of the last century, so what other
motivation do we need? :)

These notes and the accompanying presentation are an artifact of my trying to
understand what \textsl{contact geometry} is, so forgive me for not having
figured out how to explain it just yet (we are only in Section 1!).

But I can say the subject is something like how it sounds: understanding how two
(or more!) geometric objects touch, and the structure that arises from their contact.

Tangent lines and planes will be central. When two spheres touch, for example,
then at the point of contact derivatives and tangent planes coincide. These tangent
planes are examples of \textbf{contact structures}.

Also the n-truncated Taylor expansion of a function $f$ is said to be \textsl{in
  contact} with $f$.

So if we aren't inclined to agree immediately with Arnold, perhaps we can nonethless see
that the notion of \textsl{contact} isn't so foreign.

We will try to understand how \textbf{contact geometry} is more concerned with
\textbf{global} aspects of geometry, and how \textbf{locally} all contact
structures look the same.

To understand this we will develop a good bit of algebra! I hope we will find
the payoff worth it!

\subsection {The Destination}

I want to give a very brief glimpse of what it is we're going to be defining and
maybe eventually understanding.

\subsubsection {Vector Fields}

In our main text \cite{toponogov} one definition of a vector field (along a
curve) on a surface $S$ is as the derivative $\frac{d}{dt}\gamma$ where $\gamma : I \in
\mathbb{R} \to S$ is the image of a curve on the $u, v$ plane but I have got the
types a bit wrong there.

We are going to need (or at least \textsl{use}) a slightly more abstract
defintion that says: a vector field on a surface M is a linear derivation on the
set of smooth functions on M. This set is denoted $C^{\infty}(M, R)$.

So a vector field has the type: $C^{\infty}(M, R) \to C^{\infty}(M, R)$.

How does this relate to the earlier definition? And what does that definition of
a vector field imply about the definition of a \textsl{tangent vector}?

It implies a slightly different definition of that as well. It turns out that
thanks to the idea of the \textsl{directional derivative} there's a one-to-one
correspondence between tangent vectors and differential operators.

We are going to lean in to that correspondence and \textsl{define} (because who
will stop us?) tangent vectors to be linear functions from $C^{\infty}(M, R) \to
\mathbb{R}$.

\subsubsection {Integral Curves and Flows}

\subsubsection {Differential Forms}

\subsubsection {Contact}

\section {Algebra}

If calculus is in some sense about understanding nonlinear things (curves and
surfaces) in terms of linear approximations (tangent lines and spaces) then it's
not too surprising that linear algebra has a large role to play.

We should remember from calculus that the derivative is linear - indeed it is a
linear transformation of tangent spaces!

It turns out once you have this initial vector space structure you can start
unfolding even more:

\begin{description}
\item[$\cdot$ duality       ] The dual space $V^{*}$ to a given vector space $V$
\item[$\cdot$ multilinearity] Generalize from linear to \textsl{multilinear} transformations 
\end{description}

Understanding something about multilinear functions defined on the dual space
$V^{*}$ will be a central goal of these notes. These objects are called
\textsl{k-forms} and will let us define families of lines and planes which form
the basis for \textbf{contact structures}.

We will also see how the pairing of vectors and k-forms define
\textsl{invariants} that are independent of coordinates.

Let's get started!

\subsection {Co-vectors and the dual space}

We are going to assume knowledge of (finite) dimensional vector spaces over
$\mathbb{R}$ but repeat a few relevant definitions. We'll
denote a typical example by $V$ but very shortly we'll turn to a discussion of
tangent spaces like $T_{p}\mathbb{R}^{N}$.

Such spaces have \textsl{bases}: linearly independent subsets which span the
entire space. This means any element $v \in V$ can be written uniquely as the
sum of elements in the basis. The vector space $\mathbb{R}^{N}$ has the standard basis $e_{i}$
which are vectors with a $1$ in the i'th component and $0$ everywhere else.

It is a fact that a linear transformation $T : V \to W$ of vector spaces maps a
basis $v_{i}$ of $V$ to a basis $w_{j}$ of $W$.

Elements of $V$, which we usually just call vectors, could more precisely be
called \textsl{contravariant vectors}. Why? To distinguish them from
\textsl{covariant vectors}.

How is a covariant vector different from a contravariant vector? Covariant
vectors live inside the \textbf{dual space} $V^{*}$. We are going to define that
right now!

\[ V^{*} = \{ \epsilon | \epsilon : V \to \mathbb{R} \} \]

where the $\epsilon$'s are linear. So $V^{*}$ is the set of functions that map
vectors in $V$ to real numbers in $\mathbb{R}$. In fact it turns out that
$V^{*}$ is itself a vector space. So that's nice, it's like you pay for one
vector space and get one free.

Elements of $V^{*}$ are also called \textsl{linear forms} or \textsl{one-forms}.

(We haven't talked about multilinearity yet but you already know some
\textbf{fundamental} (\textsl{hint, hint}) examples of these!)

How can we tell if a given vector is secretly a covariant vector living in some
other vector space's dual space? It turns out that $V \cong (V^{*})^{*}$ so in a
sense whether or not a vector is co- or contravariant is a matter of
perspective.

What does seem to be important is this: if $v \in V$ and $\epsilon \in V^{*}$
then $\epsilon(v) \in \mathbb{R}$ is \textbf{invariant}. By that I mean it
\textsl{does not depend} on which basis you may have used to write down $v$ or
$\epsilon$ in. Coordinates don't matter.

Oh I've gotten a bit ahead of myself: since $V^{*}$ is a vector space that means
it has a basis. What does that look like? And are they related to bases for $V$?

We can associate to each basis $v_{i}$ in $V$ a \textbf{dual basis}
$\epsilon_{i}$ in $V^{*}$.

How does $\epsilon_{i}$ act on $v_{i}$? Quite simply! Each $\epsilon_{i}$
projects off the i'th component of the vector it acts on. If $V =
\mathbb{R^{N}}$ so that $v_{i} = e_{i}$ then
\begin{equation*}
  \epsilon_{i}(e_{j}) = \begin{cases}
    1 & i = j\\
    0 & i \neq j
  \end{cases}
\end{equation*}

and more generally:

\begin{equation*}
  \epsilon_{i}(v) = \epsilon_{i}(a_{1} * v_{1} + ... + a_{n} * v_{n}) = a_{1} * \epsilon_{i}(v_{1}) + ... + a_{n} * \epsilon_{i}(v_{n}) = a_{j} * \epsilon_{i}(v_{j}) = a_{i}
\end{equation*}

TODO: show that the $\epsilon_{i}$ are linearly independent and span $V^{*}$

\subsection{Forms}

A covector is the smallest example of what we will call a form: it is a 1-form.

We know at least the following forms: $dx, dy, dz$ and we know how they act on
vectors since we know they are the \textit{dual basies}.

So given any vector $v \in \mathbb{R}^{3}$ we can write $v$ as a linear
combination $xe_{1} + ye_{2} + ze_{3}$ and we have the following relationships:

TODO: talk about how $e_{k} = \frac{\partial}{\partial k}$ and use that notation everywhere!

\begin{align*}
  dx (v) = dx (xe_{i} + ye_{2} + ze_{3}) = dx (xe_{1}) + dx (ye_{2}) + dz (ze_{3}) = x \\
  dy (v) = dy (xe_{i} + ye_{2} + ze_{3}) = dy (xe_{1}) + dy (ye_{2}) + dz (ze_{3}) = y \\
  dz (v) = dz (xe_{i} + ye_{2} + ze_{3}) = dz (xe_{1}) + dz (ye_{2}) + dz (ze_{3}) = z
\end{align*}

TODO: this is pretty bad notation:

Which uses the fact that $di * \frac{\partial}{\partial j = \delta_{ij}}$ and
the linearity of the $di$.

You may have heard the term \textit{bilinear} form, which is a function that
maps two arguments (vectors in our case) to a real number. Two examples are the
first and second fundamental forms - these are also \textit{skew-symmetric}
which is another important property of forms.

So if a covector is a 1-form how do we get to (bilinear) 2-forms? 3-forms?
n-forms?

TODO: give some notation for the spaces of forms of various degree so we can
write down the type of the exterior product

We define an operation called the \textbf{exterior product} (or sometimes the
\textit{wedge product}), which is denoted $\wedge$, and somehow turns two
1-forms into a single 2-form.

We do this by relying on the \textit{determinant} since that turns out to
already be what we want: it \textit{is} an antisymmetric bilinear form!

The recipe goes like this. We have two 1-forms that we will call $\alpha$ and
$\beta$. We will thus need two vectors which we will call $a$ and $b$.

We combine them into a $2 x 2$ matrix and then take the determinant:

$
\begin{vmatrix}
  \alpha (a) & \alpha (b) \\
  \beta  (a) & \beta  (b)
\end{vmatrix}
$

So the 2-form $\gamma = \alpha \wedge \beta$ takes two vectors, applies the
1-forms $\alpha$ and $\beta$, and takes the determinant.

There is a nice geometric interpretation of the wedge product that I won't go
into right now!

But here's the TODO for this section:

1. Talk about that geometric interpretation
2. Do some concrete examples
3. Talk about the space of 3-forms and what sort of basis it has
4. Talk about integrating them, at least mention pullback!
5. Talk about volume forms!

\subsection {Differential Forms}

These are to forms (or k-forms) as vector fields are to vector fields

1. Do an example
2. Talk about how they relate to Jacobians
3. Give an example of a contact form that generates a volume form

\section {Geometry}

TODO: do I want to try to talk about integral curves / integrability?

Oh I have a section on integral curves!

If a vector field has integral curves it is integrable? But we have a field of
planes so instead of an integral curve we have an integral surface, but actually
we don't by the non-integrability condition (is that because we're a volume form
really?)

I should try to understand, so I can explain, why the non-integrability
condition is interesting!

Maybe talk about forms as actually ``sections'' of the cotangent bundle

\subsection {Plane Fields}

\subsection {Contact One-Forms}

\subsection {Contact Elements}

\subsection {Reeb Field}

\section {Generalizations}

\begin{thebibliography}{9}

\phantomsection
\addcontentsline{toc}{section}{References}
  
\bibitem{mcinerney} 
  Andrew McInerney
  \textit{First Steps in Differential Geometry: Riemannian, Contact, Symplectic}. 
  Springer-Verlag New York, 2013.

\bibitem{bachman}
  David Bachman
  \textit{A Geometric Approach to Differential Forms}.
  Birkhäuser Basel, 2006.

\bibitem{toponogov} 
  Victor Andreevich Toponogov
  \textit{Differential Geometry of Curves and Surfaces: A Concise Guide}. 
  Birkhäuser, 2006

\end{thebibliography}



\end{document}