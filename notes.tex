\documentclass{article}

\usepackage{listings, amssymb, amsmath, tikz-cd, setspace}

\usepackage{hyperref}

\usepackage{graphicx}
\graphicspath{ {./images/} }

\hypersetup{pdftex,colorlinks=true,allcolors=blue}
\usepackage{hypcap}

\begin{document}

\title {Contact Geometry}
\maketitle

\centerline{\sc \large The Presentation Notes}
% \vspace{.5pc}
% \vspace{2pc}
\onehalfspace

\tableofcontents

\newpage

\textbf{Emphatic note}: none of the mathematical ideas here represent original work in
either content or presentation! The main sources for the material are McInerney
\cite{mcinerney} and Bachman \cite{bachman} but see the full list of references for other very nice resources.

\newpage

\section {Overview}

``Contact geometry is all of geometry.'' VI Arnold.

Contact Geometry has a history going back centuries and features contributions
from Christiaan Huygens, Sophus Lie and others.

Huygens developed an approach to geometric optics in terms of wavefronts. The way that
wavefronts interact with each other leads to the notion of \textit{contact
  element}. A kind of geodesic flow associated to the contact elements via the
\textit{Reeb field} relates Huygen's principle to Fermat's principle of least time.

Lie was interested in solutions to differential equations and developed ideas
related to \textit{contact transformations}. The Legendre transformation is a
contact transformation. Contact transformations turn out to (preserve?
something?) the Reeb field. Maybe.

Physics appears to still be a central source of insight and applications -
especially to thermodynamics and the Hamiltonian formulation of classical mechanics.

This small text and accompanying presentation will mostly focus on a few of the
underlying geometric ideas, excluding contact transformations almost entirely.

Here's what we'll be talking about:

\begin{itemize}
\item contact elements
\item fields of planes and a bit about foliations
\item differential forms and the contact one-form
\item the non-integrability condition
\item the Reeb field associated to a contact one-form
\end {itemize}

\subsection {Pictures}

TODO wavefronts
TODO field of planes
TODO form / vector visualization

\includegraphics[scale=0.3]{contact_elements_bachman}

The contact structure generated by $xdy + dz$

\includegraphics[scale=0.5]{form_vis_bachman}

Visualizing forms

\pagebreak


\section {Contact Elements}

A contact element to a manifold is a hyperplane at a point of the manifold. For
all of these notes and the presentation this manifold will be $\mathbb{R}^{2}$
or $\mathbb{R}^{3}$ or subsets thereof.

Let's work in the plane $\mathbb{R}^{2}$ for a moment. Then a contact element is a point $p
\in \mathbb{R}^{2}$ and a non-vertical line-segment centered at $p$. We could
call a collection of these at every point of $\mathbb{R}^{2}$ a field of line-segments.

We need three parameters to describe this space, which we will denote
$\mathbb{CR}^{2} = \{ (x, y, m) | x, y, m \in \mathbb{R} \}$.

The $(x, y)$ components correspond to the points in the plane and the $m$
represents the slope of the line through that point.

In a little bit we will be interested in picking out certain subsets of this
space and spaces like this.

(To visualize this one can imagine a pin-wheel of planar segments, orthogonal to
$\mathbb{R}^{2}$, sitting in $\mathbb{R}^{3}$ ``above'' $\mathbb{R}^{2}$.)

\subsection {Foliations and Integrability}

If we imagine a sequence of points as being sampled from a regular curve, and the line-segments
the associated tangents, then we could integrate the tangents to recover the curve.

Usually we think of differentiating curves to get tangents, not integrating them
to get curves! Continuing in this kind of reverse direction we might call the curve an
\textit{integral curve}. I think we have seen such things a few times in class when
talking about extending some local piece of a curve on a patch to the entire surface.

If we bump up a dimension and imagine a field of planes we are led to consider
integrating those planes into an \textit{integral surface}.

Integral curves and surfaces are examples of foliations. They decompose a space
into a disjoint union of subspaces and seem pretty cool! And they certainly seem
like a cool thing to make with our contact elements!

But here's the punch line - the fields of planes we talk about in contact
geometry are \textbf{maximally} non-integrable!

This non-integrability is ensured by a condition on the gadget we use to
character our plane fields - the contact one form.

\section {Differential Forms}

Differential forms give us a way to talk about directed linear subspaces. In
particular we can use a one-form (the so-called \textbf{contact one-form}) to
characterize fields contact elements.

\subsection{Covectors and the Dual Space}
We've talked about at least three different forms in class: the first, second,
and third fundamental forms.

They are called \textit{forms} because they are functions that map vectors to a
number. The fundamental forms are also \textit{bilinear}, but there are also simpler
\textit{linear} forms. We will call these \textbf{one-forms}.

If you've taken enough linear algebra you may recognize one-forms as living in
the \textit{dual space} $\mathbb{V}^{*}$ associated to a vector space
$\mathbb{V}$. If you've taken enough physics you may recognize these as \textit{covectors}.

From a geometrical perspective a very important fact about covectors is that
they have a natural pairing with vectors. If $v \in \mathbb{V}$ and $w \in \mathbb{V}^{*}$ then the number $w(v) \in
\mathbb{R}$ is \textbf{invariant}: it does not depend on what basis $v$ or $w$
may be expressed in.

Said another way the pairing does not depend on coordinates!

Avoiding coordinate representations as much as possible is a big goal of this
formalism. Nonetheless I am about to introduce bases for both $\mathbb{V}$ and $\mathbb{V}^{*}$!

For a given basis $e_{i}$ of $\mathbb{V}$ there is an associated basis $e^{j}$
of $\mathbb{V}^{*}$ called the \textbf{dual basis}.

The dual basis has the property that under the pairing we have $e^{j}(e_{i}) = \delta^{j}_{i}$.

That is the \textbf{Kronecker delta} which is $1$ when $i = j$ and $0$
otherwise.

If we then write a $v \in \mathbb{R}^{2}$ as $v = ae_{1} + be_{2}$ and apply the
dual form $e^{1}$ we get:

\begin{align*}
  e^{1} (v) = \\
  e^{1}(ae_{1} + be_{2}) = \\
  e^{1}(ae_{1}) + e^{1}(be_{2}) = \\
  a(e^{1}(e_{1})) + b(e^{1}(e_{2})) = \\
  a * 1 + b * 0 = \\
  a
\end{align*}

This shows that elements of the dual basis are projection functions.

If we specialize to the context of differential geometry then our $\mathbb{V}$
becomes a tangent plane or space, maybe $T_{p}\mathbb{R}^{3}$. The dual vector space $\mathbb{V}^{*}$ then becomes the \textbf{cotangent space}
denoted $T_{p}^{*}\mathbb{R}^{3}$.

Remember these are just linear functions that
map tangent vectors to real numbers!

In class we're used to writing $\rho_{u}, \rho_{v}$ or maybe $r_{u}, r_{v}$ as a
basis for the tanget space.

In these notes we're going to try to use $\frac{\partial}{\partial x}, \frac{\partial}{\partial
  y}, \frac{\partial}{\partial z}$ for the tangent space and $dx, dy, dz$ for the cotangent space.

\subsection{One-forms, Two-forms, Three-forms}

Differential forms are grounded in \textit{multilinear} algebra, so we'll be
interested in \textit{multilinear} functions. But a very specific kind of
multilinear function - antisymmetric (or sometimes skew-symmetric) ones.

We actually already know an antisymmetric multilinear function and that is the
cross product!

That's going to be handy because one way to motivate the idea of (not yet
differential) forms is to try to generalize the cross product to higher
dimensions.

It turns out you can't, exactly, but you can get something even cooler: the
\textbf{wedge} (or exterior) product.

Why is it cooler? It exists for every dimension and it lets you build linear
subspaces of arbitrary rank in the same way that a cross product kind of
represents a plane.

In fact the cross product of two vectors in $\mathbb{R}^{3}$
is dual in a sense to the wedge product of two vectors. The cross product being
the normal associated to the plane while the wedge product represents the subset
of the plane spanned by the parallelogram formed by the vectors.

That was a mouthful but the picture looks like this:

TODO INSERT PICTURE

Before we go any further we want to emphasize that in this section we are
usually only working with covectors but the wedge product can operate on general
vector spaces. The point is, it makes sense to take the exterior product of any sort of vector, not just the
ones that hang out in tangent or cotangent spaces.

However since we do care mostly about dual spaces, I'll tell you that if you take
the wedge product of two one-forms you get a \textbf{two-form}. We'll look at a
calculation in a second, but in words a two-form is an antisymmetric bilinear
form.

That means it's a bilinear map $w: \mathbb{V} \times \mathbb{V} \to \mathbb{R}$
that satisfies $w(a, b) = -w(b, a)$.

Secretly you already know an antisymmetric bilinear form: the 2 x 2 determinant!
And in fact this is what we'll use to define the wedge product.

So, given two one forms $v, w$ we want to create a two-form that will map two
vectors $a, b$ to a real number. The recipe goes like this:

$
(v \wedge w) (a, b) =
\begin{vmatrix}
  v (a) & w (a) \\
  v (b) & w (b)
\end{vmatrix}
$

TODO a 2-form example

We interpret $v \wedge w$ geometrically as the parallelipiped spanned by $v$ and
$w$. I think. Does that picture make sense for two-forms or just for two-vectors
in $\mathbb{R}^{3}$? Yeah I should only talk about the geometric picture in
$\mathbb{R}^{3}$. TODO this.

An interesting consequence is that $v \wedge v = 0$ for every one-form $v$.

I should also emphasize that the antisymmetry gives the wedge product an
orientation! So these are oriented linear subspaces!

We can keep going and given three one-forms build the three-form

$
(u \wedge v \wedge \wedge w) (a, b, c) =
\begin{vmatrix}
  u (a) & v (a) & w (a) \\
  u (b) & v (b) & w (b) \\
  u (c) & v (c) & w (c) \\
\end{vmatrix}
$

for arbitrary vectors $a, b, c$.

We can keep building forms of higher degree if we keep going up a dimension but
we're going to stop here and summarize what we have.

These are:

\begin{itemize}
  \item three one-forms $dx, dy, dz$ that act on vectors $a, b, c$
  \item three two-forms $dx \wedge dy, dy \wedge dz, dz \wedge dx$ that act on
    pairs of vectors $(a, b)$
  \item one three form $dx \wedge dy \wedge dz$ that act on triples of vectors
    $(a, b, c)$
\end{itemize}

And collectively we will call these \textit{k-forms}.

In the same way that we can let a vector's coefficients vary and get a vector
field we can also let the coefficients of a k-form vary and
it is \textit{this} that we call a \textbf{differential form}.

So a differential two-form looks like $u = f dx \wedge dy$ where f is a
differential function.

\subsection {Exterior differentiation}

I won't keep you in suspense any longer! \textit{Yes}, there is a way to
differentiate a differential form!

It behaves a lot like the usual derivative, as we'll see shortly, and also has
the curious effect of bumping up the degree of a k-form. So the derivative of a
zero-form is a one-form, of a one-form is a two-form, and so on.

Oh did I talk about zero-forms before? Then now's a good time to say that a
zero-form is just a smooth function $f : \mathbb{R}^{N} \to \mathbb{R}$.

Now we can say what we want the \textbf{exterior derivative} to do to the
various forms.

\begin{itemize}
  \item on a zero-form $f$ we have the usual derivative, $df$, which is
    now a one-form that maps a tangent vector to a number
  \item on a one-form $u = f dx$ we have the two-form $du = df \wedge dx$
  \item on a two-form $v = f dx \wedge dy$ we have the three-form $dv = df \wedge dx \wedge
    dy$
  \item on a three-form we have $dw = df \wedge dx \wedge dy \wedge dz$ but
    we don't want four-forms so pretend this never happened
\end{itemize}

The exterior derivative is linear and satisfies a kind of Leibniz rule:

\begin{gather*}
  d(cv) = c * dv \\
  d(v + w) = dv + dw \\
  d(v \wedge w) = dv \wedge w + (-1)^{pq} v \wedge dw
\end{gather*}

where $p$ is the \textit{degree} (0, 1, 2, 3) of $v$ and $q$ the degree of $w$.

A \textit{very} import property: $d^{2} = 0$. This is related to the notion of
cohomology and is maybe fundamental to all physics! At least people on the
Internet will whisper that.

Now that we know about differential forms and the exterior derivative we can
talk about contact forms!

\subsection {Contact Forms}

A one-form determines a field of planes through it's \textbf{kernel}: the set of
vectors it maps to zero.

Consider $v = a dx + b dy + c dz$. If $a$ is in the kernel of $v$ then $v(a) =
0$ implies $<a, (a, b, c)> = 0$ which is a tidy way to characterize a plane.
TODO clean this up.

For a one-form $v$ to be a \textbf{contact one-form} it must also satisfy this
non-degeneracy condition:

\begin{equation}
  dv \wedge v \neq = 0
\end{equation}

The non-degeneracy condition on the contact form ensures that corresponding
field of planes will be \textit{non-integrable}, as we rather mysteriously
desire.

Real quick while I have you, are you wondering if there's a condition on the
one-form that ensures \textit{integrability}?

There is! The result is called \textbf{Frobenius' theorem} (I think there might
be a couple of those) and it says that the field of planes is integrable when

\begin{equation}
  dv \wedge v = 0
\end{equation}

So people like to say that in fact contact one-forms are \textit{maximally non-integrable}!

Okay, remember $\mathbb{CR}^{2}$? It's the space of contact elements for
$\mathbb{R}^2$. We claim that it can be represented by the one-form $dy - mdx$.

Let's check the non-degeneracy condition.

\begin{align*}
  d (dy - mdx) \wedge (dy - mdx) = \\
  -dm \wedge dx \wedge (dy - mdx) = \\
  -dm \wedge dx \wedge dy = \\
  -dx \wedge dy \wedge dm
\end{align*}

Which is the \textit{volume} form for $\mathbb{CR}^{2}$ and is non-zero.

We can also try to understand what the kernel of $w = dy - mdx$ looks like by seeing
how the form acts on a vector $x = a\frac{\partial}{\partial x} +
b\frac{\partial}{\partial y} + c\frac{\partial}{\partial z}$ when they pair to 0.

\begin{align*}
  w (x) = \\
  (dy - mdx) (a\frac{\partial}{\partial x} +
  b\frac{\partial}{\partial y} + c\frac{\partial}{\partial z}) = \\
  (dy)(a\frac{\partial}{\partial x} +
b\frac{\partial}{\partial y} + c\frac{\partial}{\partial z}) - (mdx)(a\frac{\partial}{\partial x} +
b\frac{\partial}{\partial y} + c\frac{\partial}{\partial z}) = \\
  b - ma = 0
\end{align*}

It's interesting to note that this says our slope $m$ is the ratio of the $y$
and $x$ components of our vector field.

Substituting $b = ma$ into our equation for $x$ we can find a basis for the
kernel of $w$.

\begin{align*}
  a\frac{\partial}{\partial x} +
    b\frac{\partial}{\partial y} + c\frac{\partial}{\partial z}) = \\
  a\frac{\partial}{\partial x} +
    ma\frac{\partial}{\partial y} + c\frac{\partial}{\partial z}) = \\
  a(\frac{\partial}{\partial x} +
    m\frac{\partial}{\partial y}) + c\frac{\partial}{\partial z}) 
\end{align*}

So $\ker(w) = span(\frac{\partial}{\partial x} +
    m\frac{\partial}{\partial y}, \frac{\partial}{\partial z})$ which will be a
two-dimensional subspace of the three-dimensional $\mathbb{CR}^{2}$ as we would hope!

MASSIVE TODO:

Oh, this is making me think I haven't been clear at all about what the
differential form is acting on! It is acting on elements of the contact space I
should make that clear because I \textit{think} it is correct!
  



  
The standard contact form for $\mathbb{R}^{3}$ is $xdy + dz$. We can check this
is a contact form:

\begin{align*}
  d (xdy + dz) \wedge (xdy + dz) = \\
  (dx \wedge dy + d^{2} z) \wedge (xdx + dz) = \\
  (dx \wedge dy) \wedge (xdy + dz) = \\
  (dx \wedge dy \wedge xdy) + (dx \wedge dy \wedge dz) = \\
  dx \wedge dy \wedge dz
\end{align*}

It is important to note that a given contact form does not \textit{uniquely}
determine a plane field. We can ``multiply'' $v$ by a function $f$ without
changing the kernel.

There is also a theorem of Darboux that says locally all contact forms have
``the same'' structure.

So the differences between contact structures only start to show up at the
\textit{global} level. That might make the non-integrability condition
especially interesting since we can't ``just integrate'' the contact form to understand
the global structure of the plane field!

But just because two contact forms have the same kernel \textit{doesn't} mean
they have the same Reeb field, our next and findal topic!

\section {Reeb field}

A Reeb field $R$ for a contact one-form $w$ is a vector field that satisfies the following constraints:

\begin{gather*}
  R \in \ker(dw) \\
  w(R) = 1
\end{gather*}
  
If we move (flow) our contact elements along a Reeb field then we get another
non-integrable set of contact elements. Somewhat akin to parallel transporting a
vector along a parallel vector field?

\begin{thebibliography}{9}

\phantomsection
\addcontentsline{toc}{section}{References}
  
\bibitem{mcinerney} 
  Andrew McInerney
  \textit{First Steps in Differential Geometry: Riemannian, Contact, Symplectic}. 
  Springer-Verlag New York, 2013.

\bibitem{bachman}
  David Bachman
  \textit{A Geometric Approach to Differential Forms}.
  Birkhäuser Basel, 2006.

\bibitem{toponogov} 
  Victor Andreevich Toponogov
  \textit{Differential Geometry of Curves and Surfaces: A Concise Guide}. 
  Birkhäuser, 2006

\end{thebibliography}



\end{document}